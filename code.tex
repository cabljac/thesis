\chapter{SageMath Code}
\label{app:code}
\lstset{language=Python} 

\lstset{ 
  backgroundcolor=\color{lightgray},   % choose the background color; you must add \usepackage{color} or \usepackage{xcolor}; should come as last argument
  basicstyle=\scriptsize,        % the size of the fonts that are used for the code
  frame=single
  breaklines=true
  postbreak=\mbox{\textcolor{red}{$\hookrightarrow$}\space}
}

Here we include the library of functions written to obtain the results in Chapter 3. We give a brief description of the purpose of each function, and an explanation on the nature of inputs and outputs.

First we have some auxillary functions. The function projection returns maps used to descend from a polytope \(P\) to a facet \(F\) in the recursive integration algorithm described in section~\ref{sec:barvinok}.
\begin{lstlisting}
def projection(P,F):
    A = F.as_polyhedron().equations()[0].A()
    b = F.as_polyhedron().equations()[0].b()
    A = list(A)
    B = vector(QQ,A)
    B = denominator(B)*B
    A = list(B)
    B = matrix(ZZ,B)
    K = B.right_kernel()
    if P.dimension() == 1:
        K = matrix(ZZ,[0])
    else:
        K = matrix(ZZ,K.basis())
    return [K,A]
\end{lstlisting}
The functions acoeff and bcoeff calculate specific coefficients used at points where the Barvinok algorithm terminates. They take as input a face \(F\) of a polytope \(P\), together with a vector \(c \in N_\RR\) such that Y.
\begin{lstlisting}
def acoeff(F,c):
    v_0 = F.vertices()[0].vector(); v_0
    return exp(c.dot_product(v_0))

def bcoeff(F,v):
    u_0 = F.vertices()[0].vector()
    return v.dot_product(u_0)
\end{lstlisting}
The function relvolume is simply a modification of the SageMath polytope volume method so that we may use it to calculate relative volumes of the faces of \(P\).
\begin{lstlisting}
def relvolume(P):
    if P.dimension() == 0:
        return 1
    else:
        return P.volume()
\end{lstlisting}
The function intexp calculates the integral of \(e^{\langle c, x \rangle}\) over the polyhedron P recursively using the Barvinok method. It takes as input a polyhedron \(P\), a sufficiently general vector \(L\), and a vector \(c\) for the integrand. To optimize performance values obtained at faces are cached along the way.
\begin{lstlisting}
def intexp(P,L,c,face=None, cache=None):
    if face is None: face = tuple(range(P.n_vertices()))
    if cache is None:
        cache = {}
    if cache.has_key(frozenset(face)):
        return cache[frozenset(face)]
    I = 0
    if c.is_zero():
        cache[face]= relvolume(P)
        return relvolume(P)
    else:
        for F in P.faces(P.dimension() -1):
            ProjMatrix,A = projection(P,F)
            n_F = -F.ambient_Hrepresentation()[0].A()
            n_F = n_F/gcd(n_F)
            coeff = acoeff(F,c)*(1/L.dot_product(c))*(L.dot_product(n_F))
            c_F = ProjMatrix*c
            L_F = ProjMatrix*L
            v_0 = F.vertices()[0].vector(); v_0
            Vert = [ProjMatrix.transpose().solve_right(v.vector() - v_0)  for v in F.vertices() ]
            face_F = tuple([face[i] for i in F._ambient_Vrepresentation_indices])
            P_F = Polyhedron(vertices = Vert, backend="cdd")
            I = I + coeff*intexp(P_F,L_F,c_F,face_F,cache)
    cache[frozenset(face)]=I
    return I
\end{lstlisting}

The function intxexp calculates the integral of \(\langle x,v \rangle \cdot e^{\langle c,x \rangle}\) over the polyhedron \(P\) recursively using the Barvinok method. The input is a polyhedron \(P\),
a sufficiently general vector \(L\), and vectors \(c\) and \(v\) for the integrand. To optimize performance values obtained at faces are again cached along the way.
\begin{lstlisting}
def intxexp(P,L,c,v,face=None,cache=None,cache2=None):
    if face is None: face = tuple(range(P.n_vertices()))
    if cache2 is None:
        cache2 = {}
    if cache is None:
        cache = {}
    if cache.has_key(frozenset(face)):
        return cache[frozenset(face)] 
    I = 0
    if c.is_zero():
        if v.is_zero() : return 0
        for T in list(P.triangulate()):
            T = [P.Vrepresentation()[ZZ(t)].vector() for t in T]
            bary = sum(T)/len(T)
            T = Polyhedron(vertices = T)
            I = I + T.volume()*bary.dot_product(v)
    else:
        for F in P.faces(P.dimension() -1):
                ProjMatrix = projection(P,F)
                L_F  = ProjMatrix[0].insert_row(0,vector(ZZ,ProjMatrix[1])).transpose().solve_right(L)[1:]
                c_F = ProjMatrix[0]*c
                v_F = ProjMatrix[0]*v
                v_0 = F.vertices()[0].vector(); v_0
                Vert = [ProjMatrix[0].transpose().solve_right(w.vector() - v_0)  for w in F.vertices() ]
                face_F = tuple([face[i] for i in F._ambient_Vrepresentation_indices])
                P_F = Polyhedron(vertices = Vert,backend="cdd")
                n_F = -F.ambient_Hrepresentation()[0].A()
                n_F = n_F/gcd(n_F)
                I = I + (L.dot_product(n_F)/L.dot_product(c))*acoeff(F,c)*(intxexp(P_F,L_F,c_F,v_F,face_F,cache,cache2) + bcoeff(F,v)*intexp(P_F,L_F,c_F,face_F,cache2)) - L.dot_product(n_F)*(v.dot_product(L)/(L.dot_product(c)^2))*acoeff(F,c)*intexp(P_F,L_F,c_F,face_F,cache2)
    cache[frozenset(face)]=I
    return I
\end{lstlisting}

The function degenerations returns the list of degeneration polytopes (normal or otherwise) of a \(T\)-variety specified by a divisorial polytope. It takes inputs \(L\) and \(B\), where \(L\) is a list of matrices and \(B\) is the base polytope, referred to as \(\Box\) in this thesis. Each matrix in \(L\) represents a piecewise affine function on \(B\) in the following way: Suppose \(v_j\) are the columns of \(M\) in the list \(L\). We then form the piecewise affine function:
\[
\psi_M(x_1,\dots,x_n) := \min_j \{ (1,x_1,x_2,\dots,x_n) \cdot v_j \}
\]
\begin{lstlisting}
def degenerations(L,B):
    dim = L[0].nrows()
    genericfunction = matrix(QQ, dim, 1, lambda i, j: 0)
    L = L +[genericfunction]
    def shift(A,r):
        shift = matrix(QQ,  A.transpose().nrows() , 1, lambda i, j: r);
        zeroes = matrix(QQ, A.transpose().nrows(), A.nrows() -1 , lambda i, j: 0);
        return A + shift.augment(zeroes).transpose()
    def minsum(L):
        msCurrent = L[0]
        def minsum_step(A1,A2):
            C = matrix(QQ,A2.nrows(),1,lambda i, j: 0)
            for c in A1.columns():
                for d in A2.columns():
                    C = C.augment(d+c);
            return C.delete_columns([0])
        for l in L[1:]:
            msCurrent = minsum_step(msCurrent,l)
        return msCurrent
    def degenerations_step(L,k,B):
        up = shift(L[k],1)
        base = matrix(QQ, B.inequalities_list()[:]);
        low = L[:]; low.pop(k);
        low = minsum(low).transpose();
        up = up.transpose()
        base = base.augment(matrix(QQ,  base.nrows() , 1, lambda i, j: 0)).transpose()
        up = up.augment(matrix(QQ, up.nrows(), 1, lambda i, j: -1)).transpose()
        low = low.augment(matrix(QQ, low.nrows(), 1, lambda i, j: 1)).transpose()
        low = shift(low,1)
        P = Polyhedron(ieqs = [list(v) for v in up.augment(low.augment(base)).columns()])
        return P;
    Polyhedra = [];
    for k in range(len(L)):
        Polyhedra = Polyhedra + [degenerations_step(L,k,B)];
    return Polyhedra
\end{lstlisting}
