\documentclass[12pt,twoside]{muthesis}

  %% this block is for solitons tables

\usepackage{listings}


\usepackage{multirow}



\usepackage{float}

\usepackage{amsmath}
\usepackage{amsthm}
\usepackage{cite}
\usepackage{amscd,amssymb,amsmath,nccmath,microtype}
\usepackage{calligra,mathrsfs}
\usepackage{graphicx}
\usepackage[hyphens]{url}
\usepackage{hyperref}
\usepackage{changepage}
\usepackage{caption}
\usepackage{listings}
\usepackage{amssymb}
\usepackage{amsfonts}
\usepackage{nicefrac}
\usepackage{stmaryrd}

\usepackage[hyphens]{url}
\usepackage{hyperref}
%\PassOptionsToPackage{gray}{xcolor}
\usepackage{xcolor}
\usepackage{colortbl}
\usepackage[most]{tcolorbox}
\usepackage{booktabs}
\usepackage{longtable}

\usepackage{xypic}

\usepackage{tikz}
\usepackage{multirow}
\usepackage{hyperref}
\usepackage{subcaption} 
\usepackage{mathtools}
\usepackage[draft,multiuser,layout={margin,index}]{fixme}
\usepackage{pifont}


\usepackage{tikz-cd}

\usepackage{tikz-3dplot}
\tdplotsetmaincoords{70}{110}

\usepackage{subcaption}

\usetikzlibrary{shapes.geometric}
\usepackage{mathrsfs}
\usetikzlibrary{arrows}



\makeatletter
\newcommand*{\@rowstyle}{}

\newcommand*{\rowstyle}[1]{% sets the style of the next row
  \gdef\@rowstyle{#1}%
  \@rowstyle\ignorespaces%
}

\newcolumntype{=}{% resets the row style
  >{\gdef\@rowstyle{}}%
}

\newcolumntype{+}{% adds the current row style to the next column
  >{\@rowstyle}%
}

\makeatother


\newtheorem*{ttheorem}{Tian's criterion}
\newtheorem{theorem}{Theorem}
\newtheorem{corollary}{Corollary}
\newtheorem{construction}{Construction}
\newtheorem{definition}{Definition}
\newtheorem{example}{Example}
\newtheorem{remark}{Remark}
\newtheorem{lemma}{Lemma}
\newtheorem{proposition}{Proposition}


\newcommand{\cmark}{\ding{51}}%
\newcommand{\xmark}{\ding{55}}%
\newcommand{\pt}{\{\mathbf{pt}\}}
\renewcommand{\Vert}{\mathcal{V}}
\newcommand{\Hor}{\mathcal{H}}
%\newcommand{\QGIT}{/^{^{\mathsf{GIT}}}}
\newcommand{\QGIT}{{\!/\!\!/}}
\newcommand{\tXn}{{\widetilde{X}^\circ}}
\newcommand{\Xn}{{X^\circ}}
\newcommand{\tX}{{\widetilde{X}}}
\newcommand{\tD}{{\widetilde{D}}}
\newcommand{\tL}{{\widetilde{\L}}}
\newcommand{\T}{{\mathcal{T}}}
\newcommand{\tY}{{\widetilde{Y}}}
\newcommand{\CC}{\mathbb{C}}
\renewcommand{\L}{\mathcal{L}}
\newcommand{\RR}{\mathbb{R}}
\newcommand{\B}{\mathcal{B}}
\newcommand{\QQ}{\mathbb{Q}}
\newcommand{\ZZ}{\mathbb{Z}}
\newcommand{\NN}{\mathbb{N}}
\newcommand{\PP}{\mathbb{P}}
\newcommand{\A}{\mathbb{A}}
\newcommand{\m}{\mathfrak{m}}
\newcommand{\lin}[1]{\text{span}\,(#1)}
\newcommand{\equivn}{{\stackrel{\scriptscriptstyle\text{num}}{\sim}}}
\newcommand{\KK}{k}
\newcommand{\xx}{\mathrm{x}}

\newcommand{\savefootnote}[2]{\footnote{\label{#1}#2}}
\newcommand{\repeatfootnote}[1]{\textsuperscript{\ref{#1}}}

\newcommand{\fan}{\Xi}
\newcommand{\CO}{{\mathcal{O}}}
\newcommand{\f}{{\mathfrak{f}}}
\newcommand{\h}{{\mathfrak{h}}}
\newcommand{\cA}{{\mathcal{A}}}
\newcommand{\X}{{\mathcal{X}}}
\newcommand{\Y}{{\mathcal{Y}}}
\newcommand{\Z}{{\mathcal{Z}}}
\newcommand{\D}{{\mathfrak{D}}}
\newcommand{\R}{{\mathcal{R}}}
\newcommand{\bphi}{\bar \Phi}
\newcommand{\sufficient}{valuable }
\renewcommand{\O}{\mathcal{O}}
\renewcommand{\div}{\text{div}}
\DeclareMathOperator{\TV}{TV}
\DeclareMathOperator{\barycenter}{bary}
\DeclareMathOperator{\DF}{DF}
\DeclareMathOperator{\Sl}{Sl}
\DeclareMathOperator{\GIT}{GIT}
\DeclareMathOperator{\GL}{GL}
\DeclareMathOperator{\Bl}{Bl}
\DeclareMathOperator{\Loc}{Loc}
\DeclareMathOperator{\ray}{ray}
\DeclareMathOperator{\codim}{codim}
\DeclareMathOperator{\NE}{NE}
\DeclareMathOperator{\face}{face}
\DeclareMathOperator{\quot}{Quot}
\DeclareMathOperator{\rk}{rank}
\DeclareMathOperator{\relint}{relint}
\DeclareMathOperator{\tail}{tail}
\DeclareMathOperator{\SF}{SF}
\DeclareMathOperator{\tcadiv}{T-CaDiv}
\DeclareMathOperator{\tdiv}{T-Div}
\DeclareMathOperator{\PD}{PD}
\DeclareMathOperator{\im}{im}
\DeclareMathOperator{\coker}{coker}
\DeclareMathOperator{\Hom}{Hom}
\DeclareMathOperator{\divisor}{div}
\DeclareMathOperator{\cadiv}{CaDiv}
\DeclareMathOperator{\wdiv}{Div}
\DeclareMathOperator{\pic}{Pic}
\DeclareMathOperator{\spec}{Spec}
\DeclareMathOperator{\proj}{Proj}
\DeclareMathOperator{\ord}{ord}
\DeclareMathOperator{\supp}{supp}
\DeclareMathOperator{\Pol}{Pol}
\DeclareMathOperator{\loc}{Loc}
\DeclareMathOperator{\SPEC}{\mathbf{Spec}}
\DeclareMathOperator{\conv}{conv}
\DeclareMathOperator{\pos}{pos}
\DeclareMathOperator{\vol}{vol}
\DeclareMathOperator{\cl}{Cl}
\DeclareMathOperator{\num}{NS}
\DeclareMathOperator{\syz}{Rel}
\DeclareMathOperator{\Aut}{Aut}
\DeclareMathOperator{\lct}{g\bf{lct}}
\DeclareMathOperator{\init}{in}
\DeclareMathOperator{\numq}{NS_{\hspace{.2pt}\QQ}}
\DeclareMathOperator{\Div}{div}
\DeclareMathOperator{\Cox}{Cox}
\DeclareMathOperator{\Cl}{Cl}
\DeclareMathOperator{\Ric}{Ric}
\DeclareMathOperator{\PGL}{PGL}
\DeclareMathOperator{\glct}{glct}


\providecommand{\Ric}{\mathop{\rm Ric}\nolimits}
\providecommand{\bc}{\mathop{\rm bc}\nolimits}
\providecommand{\Aut}{\mathop{\rm Aut}\nolimits}
\providecommand{\DF}{\mathop{\rm DF}\nolimits}

\providecommand{\inte}{\text{int}}
\renewcommand{\L}{\mathcal{L}}
\providecommand{\vol}{\mathop{\rm vol}\nolimits}

\def\diagram1{

\definecolor{wrwrwr}{rgb}{0.3803921568627451,0.3803921568627451,0.3803921568627451}
\definecolor{rvwvcq}{rgb}{0.08235294117647059,0.396078431372549,0.7529411764705882}

\begin{tikzpicture}[scale=0.4,line cap=round,line join=round,>=triangle 45,x=1cm,y=1cm]
\clip(-8.69388785352341,-6.85149354395804) rectangle (12.217649006953684,5.977670174126094);
\fill[line width=0.4pt,color=rvwvcq,fill=rvwvcq,fill opacity=0.10000000149011612] (3.95,-3.22) -- (-4.499755258769425,-0.465188547516475) -- (-4.021306091512521,1.7101502921524365) -- (3.8019942792933907,3.9467619721580767) -- (9.206358516755902,2.6979556386504635) -- (7.16248304580607,-1.3101637913940103) -- cycle;
\draw [line width=0.4pt,color=wrwrwr] (-4.499755258769425,-0.465188547516475)-- (-4.021306091512521,1.7101502921524365);
\draw [line width=0.4pt,color=wrwrwr] (-4.021306091512521,1.7101502921524365)-- (3.8019942792933907,3.9467619721580767);
\draw [line width=0.4pt,color=wrwrwr] (-4.499755258769425,-0.465188547516475)-- (3.95,-3.22);
\draw [line width=0.4pt,color=wrwrwr] (2.4376800090649002,0.7867993541259463)-- (-16.628346540838827,-10.065412601807587);
\draw [line width=0.4pt,color=rvwvcq] (3.95,-3.22)-- (-4.499755258769425,-0.465188547516475);
\draw [line width=0.4pt,color=rvwvcq] (-4.499755258769425,-0.465188547516475)-- (-4.021306091512521,1.7101502921524365);
\draw [line width=0.4pt,color=rvwvcq] (-4.021306091512521,1.7101502921524365)-- (3.8019942792933907,3.9467619721580767);
\draw [line width=0.4pt,color=rvwvcq] (3.8019942792933907,3.9467619721580767)-- (9.206358516755902,2.6979556386504635);
\draw [line width=0.4pt,color=rvwvcq] (9.206358516755902,2.6979556386504635)-- (7.16248304580607,-1.3101637913940103);
\draw [line width=0.4pt,color=rvwvcq] (7.16248304580607,-1.3101637913940103)-- (3.95,-3.22);
\draw [line width=0.4pt,color=wrwrwr] (-4.499755258769425,-0.465188547516475)-- (-6.0703572252974505,-0.590007254352775);
\draw [line width=0.4pt,color=wrwrwr,domain=-8.69388785352341:12.217649006953684] plot(\x,{(-7.125388691254146-1.5706019665280255*\x)/0.12481870683630003});
\draw [line width=0.4pt,color=wrwrwr] (-0.2748776293847124,-1.8425942737582377)-- (-0.6233328024886629,-2.9114009714819993);
\draw (2.430829827729521,1.0842605845140028) node[anchor=north west] {$b$};
\draw (-4.2953174358945745,-2.746161154171117) node[anchor=north west] {$p_w$};
\draw (-2,-1.4) node[anchor=north west] {$q$};
\draw (-0.65,-2.471250503069314) node[anchor=north west] {$n$};
\draw (-8.3,4) node[anchor=north west] {$\Pi(w,a)$};
\draw [->,line width=0.05pt,color=wrwrwr] (-0.2748776293847124,-1.8425942737582377) -- (-0.6233328024886629,-2.9114009714819993);
\draw [->,line width=0.05pt,color=wrwrwr] (-4.499755258769425,-0.465188547516475) -- (-6.0703572252974505,-0.590007254352775);
\draw (-6.3296562540479115,-0.505) node[anchor=north west] {$w$};
\draw (-0.20831242284778076,2.0372841750002526) node[anchor=north west] {$\Huge{P}$};
\draw (-0.4648956972094629,-0.5652033220968145) node[anchor=north west] {$O$};
\draw (-4.533573333516137,0.2) node[anchor=north west] {$a$};
\begin{scriptsize}
\draw [fill=wrwrwr] (-4.499755258769425,-0.465188547516475) circle (1.5pt);
\draw [fill=wrwrwr] (-0.4676952405849731,-0.8669142782450808) circle (1.5pt);
\draw [fill=wrwrwr] (2.4376800090649002,0.7867993541259463) circle (1.5pt);
\draw [fill=wrwrwr] (-4.294715291030033,-3.0452198962684096) circle (2pt);
\draw [fill=wrwrwr] (-1.4873591410315417,-1.4472978596114847) circle (1.5pt);
\end{scriptsize}
\end{tikzpicture}
}