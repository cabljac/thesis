\chapter{Conclusions and further work}

In this brief final chapter we summarize the work in this thesis and describe how one might pursue further related results. As outlined in the introduction to this thesis, we set out with a goal of pushing further the understanding of canonical metrics on Fano \(T\)-varieties, and constructing examples as evidence of equivariant methods in this context.

We achieved this goal. Our results on K\"ahler-Ricci solitons in Chapter~\ref{chap:sol} provided new examples, and give new insights into modified \(K\)-stability in complexity one. In Chapter~\ref{chap:R(X)} we give an effective formula for an invariant defined precisely in terms of the existence of certain canonical metrics. Finally, we provide new examples of K\"ahler-Einstein metrics in complexity two, showing that equivariant methods are not restricted to actions of low complexity.
\subsection*{Further work}
\subsubsection*{K\"ahler-Ricci solitons}
All Fano toric orbifolds admit a K\"ahler-Ricci soliton in \cite{shi2012kahler}. However, as shown in \cite{cable2018classification}, not all complexity one Fano \(T\)-varieties do. Nevertheless all \textit{smooth} examples so far have admitted a K\"ahler-Ricci soliton. This obviously leads to the question of whether it holds true in general. On the one hand it seems like the same approach as the toric case cannot work, as the proof for toric orbifolds seems very similar to the proof in the smooth case. The first step would be to calculate more examples and see if there exists a counterexample in higher dimensions.
\subsubsection*{R(X)}
It is interesting that only special test configurations with negative classical Donaldson-Futaki character contribute to the infimum (\ref{eq:R(X)inf}) for complexity one \(T\)-varieties, and it would be of interest to see if this held true in any more general context. If more divisorial polytopes are calculated for \(T\)-varieties in higher dimensions, \(R(X)\) is easy to calculate from Theorem~\ref{thm:R(X)}.
\subsubsection*{General arrangment varieties}
The proof of the results in Chapter~\ref{chap:gav} relies on the bound for \( \glct_{S_4}(\PP^2, B_\gamma)\). This is the only obstruction in extending the results to higher dimensonal elements of the families: Taking limits along a \(\CC^*\)-action in higher dimensions is not useful, as the resulting divisor may be more complicated. In one dimension higher, for example, the most we are able to say about the components of such a limit is that they are ruled surfaces intersecting at least at one common point.

With more knowledge or methods of calculating such alpha invariants in higher dimensions, we would obtain more K\"ahler-Einstein metrics and in particular they would be in higher complexity.